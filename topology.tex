\documentclass{report}

\usepackage{cmap}
\usepackage[T2A]{fontenc}
\usepackage[utf8]{inputenc}
\usepackage[english, russian]{babel}
\usepackage{mathtools}
\usepackage{amsfonts}
\usepackage{amsthm}

\title{Topology}
\author{Tigran Alvandian}
\date{February 2022}

\begin{document}

\maketitle 

\chapter{Топология прямой}
\section{Вступительное слово}


На множестве действительных чисел $\mathbb{R}$  имеются следующие структуры: \\ 
\begin{itemize}

\item \textit{Структура поля}, то есть на нём заданы операция сложения $+ \colon \mathbb{R} \times \mathbb{R} \to \mathbb{R}$, операция умножения $\cdot: \mathbb{R} \times \mathbb{R} \to \mathbb{R}$, а также выбраны специальные элементы $0, 1 \in \mathbb{R}$. Вместе эти операции и элементы удовлетворяют набору определённых условий, знакомых вам со школы: ассоциативность, коммутативность, дистрибутивность, обратимость всех элементов относительно обеих операций, кроме ноля по умножению (аксиомы поля).\\
\item На множестве $\mathbb{R}$ введено отношение порядка $\leqslant$, согласующееся со структурой поля («действительные числа образуют упорядоченное поле»). \\
\end{itemize}
Отметим два важных свойства $\mathbb{R}$: \\ 

\begin{itemize}
\item На $\mathbb{R}$ выполнена аксиома Архимеда: $\forall a \in \mathbb{R}$ существует $n \in \mathbb{N}$ такое, что $n \geq a$. \\
\item Множество $\mathbb{R}$ обладает свойством полноты, то есть у любого его непустого ограниченного сверху подмножества имеется точная верхняя грань. \\
\end{itemize}

Напомним определение точной верхней грани.

\textbf{Определение}. Пусть $X \subset \mathbb{R}$ — ограниченное множество. Число $m \in \mathbb{R}$ называется верхней гранью множества $X$, если $\forall x \in X$ выполнено, что $x \leq m$. Число $m_0$ называется точной верхней гранью множества $X$, если оно является его верхней гранью и для любой другой верхней грани $m$ выполнено $m_0 \leq m$. \\ 

Примером упорядоченного поля, не обладающего свойством полноты, является поле рациональных чисел $\mathbb{Q}$. \\ 

Полнота $\mathbb{R}$ является следствием из \texit{принципа вложенных отрезков}.

Теорема («Принцип вложенных отрезков»). Пусть $I_n$ — последовательность отрезков в $\mathbb{R}$, причём $I_{n+1} \subseteq I_n$. Тогда $\bigcap_n I_n \neq \varnothing$. При этом если последовательность длин отрезков $|I_n|$ стремится к нулю, то $\bigcap_n I_n$ состоит из одной точки. \\ 

Архимедово упорядоченное поле полно тогда и только тогда, когда в нём выполнен принцип вложенных отрезков. \\ 

\textbf{Подмножества действительной прямой} \\ 

Мы встречались с разными подмножествами прямой (промежутки и их объединения, рациональные числа и так далее).\\

Некоторые из этих подмножеств состоят из разрозненных «кусков», а некоторые, наоборот, являются связными (что это значит, мы узнаем чуть позже). На некоторых из них можно задать неограниченные непрерывные функции, а на других нельзя. Это всё топологические свойства подмножеств прямой. Прежде чем изучать их основательно, посмотрим на ещё один пример подмножества прямой: канторово множество.

\section{Канторово множество}

\textbf{Построение канторова множества} \\ \\ 
Опишем индуктивное построение стандартного канторова множества.\\
Рассмотрим единичный отрезок $[0,1]$. На первом шаге разобьём его на три равных отрезка $\left[0;\frac{1}{3}\right]$, $\left[\frac{1}{3}; \frac{2}{3}\right]$, $\left[\frac{2}{3} ;1\right]$. «Выкинем» среднюю часть (вернее, соответствующий интервал $( \frac{1}{3}; \frac{2}{3} )$), при этом останутся два крайних отрезка $\left[0 ; \frac{1}{3}\right]$ и $\left[\frac{2}{3} ; 1\right]$.\\ \\
Будем повторять эту процедуру бесконечное число раз: на $k$-м шаге из каждого отрезка, оставшегося после $(k-1)$-го шага, выкинем средний интервал.\\

\photo{https://edu.sirius.online/noo-back/content/_image/2cc581b995ec37c9d2955b168fd7bf3de6d47840} \\ 

Множество, которое получится в итоге, называется стандартным канторовым множеством. Будем обозначать его $K$.\\

\textbf{Мощность канторова множества} \\
Заметим, что канторово множество не пусто, то есть в процессе выкидывания середин отрезков мы не выкинули вообще все точки отрезка: действительно, любой из концов изначального отрезка $[0,1]$ не будет выкинут ни на каком шаге, а значит, будет принадлежать множеству $K$. Более того, концы любого из построенных на каждом шаге отрезков будут принадлежать $K$. Итак, $K$ по меньшей мере счётно.\\ 

Докажем, что $K$ имеет мощность континуум.\\

\textbf{Теорема}. Существует взаимно однозначное соответствие между точками канторова множества и бесконечными последовательностями из $0$ и $2$.\\

\begin{proof} Коротко говоря, достаточно рассмотреть троичную запись координат чисел на отрезке $[0,1]$: точки $K$ имеют координаты, в записи которых отсутствуют $1$.\\

Более подробно: рассмотрим некоторую точку канторова множества. На каждом шаге построения $K$ она лежит в одном из построенных отрезков. При переходе к следующему шагу надо уточнить, лежит ли она в левом или в правом подотрезке. Поставим каждой точке в соответствие последовательность нулей и двоек следующим образом: если на первом шаге точка лежит в правой трети, то запишем в последовательность $0$, а если в левой — запишем $2$. На втором шаге построения запишем второй знак последовательности и так далее: если на $k$-м шаге точка оказалась в правой трети отрезка, запишем на $k$-м месте последовательности $0$, если же в левой — запишем в последовательность $2$.\\

\photo{https://edu.sirius.online/noo-back/content/_image/7f5a69f5daf090815a6e7d697919888489c642e0}\\

На каждом шаге конструкции построения каждая точка из канторова множества лежит в некотором отрезке из построения, причём отрезки, полученные на одном и том же шаге, не пересекаются. Следовательно, каждой точке $K$ ставится в соответствие ровно одна последовательность нулей и двоек. В обратную сторону: каждая последовательность нулей и двоек определяет собой последовательность вложенных отрезков с длинами, убывающими втрое на каждом шаге. Такая последовательность отрезков имеет единственный общий элемент. \\
\end{proof}

\textbf{Вариации}\\

Можно построить и другие варианты канторова множества. Например, можно выкидывать на каждом шаге не одну треть, а одну пятую из каждого отрезка. Также можно на каждом шаге варьировать выкидываемую долю. Например, на первом шаге выкинуть одну треть от рассматриваемого отрезка, на втором — одну девятую от каждого рассматриваемого отрезка, на третьем — одну двадцать седьмую долю и так далее. Позже при изучении теории меры можно будет вычислить меру остающегося канторова множества.\\

\textbf{Салфетка Серпинского}\\

Подобную индуктивную процедуру можно применить не только к отрезку, но и, например, к подмножествам плоскости.


Рассмотрим замкнутую область на плоскости, ограниченную треугольником. Отметим середины сторон треугольника и соединим их отрезками. Они разбивают исходную область на 4 подобных ей треугольных области. На первом шаге выкинем серединный треугольник, а дальше будем шаг за шагом повторять эту процедуру для каждого остающегося треугольника.\\

Множество, которое получится в итоге, называется салфеткой Серпинского.\\

Подобную процедуру можно также проводить с тетраэдром или другими объёмными объектами. \\

\photo{https://edu.sirius.online/noo-back/content/_image/248af0a9bdf7539e133d46d44237f5762c8da97b}\\

\section{Открытые замкнуты подмножества прямой}

\textbf{Открытые множества}

\textbf{Определение 1}. Множество $U \subset \mathbb{R}$ называется открытым, если для любой точки $x \in U$ найдётся $\varepsilon > 0$ такое, что $(x-\varepsilon; x+\varepsilon) \subseteq U$. Иными словами, каждый элемент $U$ содержится в $U$ вместе с некоторым интервалом, содержащим его.\\

\textbf{Определение 2}. Точка $x$ называется внутренней точкой множества $U$, если найдётся $\varepsilon > 0$ такое, что $(x-\varepsilon; x+\varepsilon) \subseteq U$.\\

Открытые множества — это в точности те множества, у которых всякая точка является внутренней.

Отрезок $[a,b]$ не является открытым: его концы — не внутренние точки. А вот интервал $(a,b)$ — открытое множество. Также открытым множеством является луч без граничной точки $(x; +\infty)$.


\textbf{Свойства открытых множеств}:
\begin{itemize}
\item $\mathbb{R}$ открыто, $\emptyset$ открыто.
\item Объединение любого числа открытых множеств открыто.
\item Пересечение конечного числа открытых множеств открыто.
\end{itemize}

\begin{proof}\\

Докажем первое свойство. Открытость пустого множества тривиальна. Открытость $\mathbb{R}$ тоже почти очевидна: просто возьмём в качестве окрестности точки $x$ интервал $(x-1, x+1)$.\\

Разберём доказательство второго свойства. Пусть $\{U_{\alpha}\}$ — произвольный набор открытых множеств и $U = \bigcup_{\alpha} U_{\alpha}$. Возьмём произвольную точку $x \in U$. Существует открытое множество $U_{\alpha_0} \in \{U_{\alpha}\}$, которое содержит точку $x$. Так как $U_{\alpha_0}$ открыто, оно содержит $x$ вместе с некоторым интервалом $(x-\varepsilon; x+\varepsilon)$. Получаем, что $x \in (x-\varepsilon; x+\varepsilon) \subseteq U_{\alpha_0} \subseteq U$, поэтому $U$ тоже содержит $x$ вместе с интервалом $(x-\varepsilon; x+\varepsilon)$. \\

Осталось проверить третье свойство. Пусть есть набор открытых множеств $U_1, \ldots, U_n$ и $x \in \bigcap_n U_n$. Поскольку $x$ — внутренняя точка для каждого множества $U_i$, она содержится вместе с интервалом $(x-\varepsilon_1; x+\varepsilon_1)$ во множестве $U_1$, вместе с интервалом $(x-\varepsilon_2; x+\varepsilon_2)$ во множестве $U_2$ и т.д. Без ограничения общности можно считать, что $\varepsilon_1 \geq \varepsilon_2 \geq \ldots \varepsilon_n>0$. Пересечение конечного числа интервалов $\bigcap_j (x-\varepsilon_j; x+\varepsilon_j)$ не пусто и совпадает с наименьшим интервалом $(x-\varepsilon_n; x+\varepsilon_n)$. Значит, $(x-\varepsilon_n; x+\varepsilon_n) \subseteq \bigcap U_n$.\\
\end{proof}

Бесконечное пересечение открытых множеств не обязательно открыто. Например, $\bigcap_n \left(-\frac{1}{n};\frac{1}{n} \right) = \{0\},$ а это множество не является открытым.\\

\textbf{Замкнутые множества}.\\ 

\textbf{Определение 3}. Проколотой $\varepsilon$-окрестностью точки $x \in \mathbb{R}$ называется множество $\mathring{U}_{\varepsilon}(x)=(x-\varepsilon; x+\varepsilon)\setminus\{x\}$.\\ 

\textbf{Определение 4}. Точка $x \in \mathbb{R}$ называется предельной для множества $M \subset \mathbb{R}$, если $\forall \varepsilon > 0$ множество $\mathring{U}_{\varepsilon}(x) \cap M$ не пусто.\\ 

Если в определении предельной точки заменить проколотую окрестность на обычную, получится не очень полезное определение. Например, ему будет удовлетворять каждая точка множества $M$. Для сравнения, при правильном определении точка $0$ не является предельной для множества $\mathbb{N}$, поскольку $\mathring{U}_{1/2}(0)$ не содержит точек из $\mathbb{N}$. Точки с таким свойством будем называть изолированными. \\

\textbf{Определение 5}. Точка $x$ множества $M$ называется изолированной точкой множества, если $\exists~\varepsilon >0$ такой, что $\mathring{U_{\varepsilon}}(x) \cap M = \emptyset$.\\

\textbf{Определение 6}. Подмножество прямой называется замкнутым, если оно содержит все свои предельные точки.\\

Примеры замкнутых множеств: пустое множество $\varnothing$, вся прямая $\mathbb{R}$, отрезок $[a,b]$, одноточечное множество $\{x\}$, множество натуральных чисел $\mathbb{N}$. \\

Интервал $(a,b)$ — не замкнутое множество: его края являются его предельными точками и не принадлежат ему. Рациональные числа тоже не замкнуты: множество предельных точек $\mathbb{Q}$ — это вся прямая (такие множества называются всюду плотными). Множество {$\frac{1}{n} | n \in \mathbb{N}$} не замкнуто, но становится замкнутым, если к нему добавить $0$. \\

Свойства замкнутых множеств: 
\begin{itemize}
\item $\emptyset$ замкнуто, $\mathbb{R}$ замкнуто. 
\item Любое пересечение замкнутых множеств замкнуто. 
\item Объединение конечного числа замкнутых множеств замкнуто.
\end{itemize}

\begin{proof}
Докажем первое свойство. Для любого $x \in \mathbb{R}$, для $\forall \varepsilon > 0$ множество $\mathring{U_{\varepsilon}}(x) \cap \mathbb{R}$ не пусто. Действительно, достаточно взять $n> \frac{1}{\varepsilon}$ и убедиться, что точка $x + \frac{1}{n}$ содержится в этом множестве. Значит, каждый элемент $\mathbb{R}$ является предельной точкой множества $\mathbb{R}$, то есть множество действительных чисел содержит все свои предельные точки. Замкнутость пустого множества также очевидна. \\

Разберём доказательство второго свойства. Пусть $\{F_{\alpha}\}$ — произвольный набор замкнутых множеств и $F = \bigcap_{\alpha} F_{\alpha}$. Возьмём предельную точку $x$ множества $F$. Мы знаем, что $\forall \varepsilon > 0$ множество $\mathring{U_{\varepsilon}} \cap F$ не пусто. В частности, для $\forall \alpha$ множество $\mathring{U_{\varepsilon}} \cap F_{\alpha}$ не пусто. Тогда точка $x$ будет являться предельной точкой для каждого множества $F_{\alpha}$. Так как множества $F_{\alpha}$ замкнутые, точка $x$ будет принадлежать каждому из множеств, а значит и принадлежать их пересечению. \\

Осталось проверить третье свойство. Пусть есть набор замкнутых множеств $F_1, \ldots, F_n$ и $F =\bigcup_n F_n$. Возьмём предельную точку $x$ множества $F$. Предположим, что она не является предельной точкой ни для какого $F_i$, $i=1,\cdots,n$. Тогда существуют $\varepsilon_1 >0, \cdots, \varepsilon_n >0$ такие, что $\mathring{U_{\varepsilon_i}}(x) \cap F_i = \varnothing$, $i=1,\cdots,n$. Возьмём $\varepsilon=min\{\varepsilon_i|i=1,\cdots,n\}$. Тогда $\mathring{U_{\varepsilon}}(x) \cap \bigcup_n F_n = \varnothing$, следовательно $x$ — не предельная точка множества $F$. Получили противоречие, а значит, точка $x$ — предельная для некоторого множества $F_i$. Так как $F_i$ замкнуто, отсюда следует, что $x \in F_i$ и, соответственно, $x \in F$.\\

\end{proof}

Бесконечное объединение замкнутых множеств не обязательно замкнуто: объединение счётного числа одноточечных множеств $\bigcup_{n \in \mathbb{N}}\left\{\frac{1}{n} \right\}$ не замкнуто, так как оно не содержит свою предельную точку $0$.\\

\textbf{Двойственность между открытыми и замкнутыми множествами}.\\

Заметим, что перечисленные свойства откртытх и замкнутых множеств двойственны:

\begin{tabular}{|l|l|}\hline
\ \ \ \ \ \textbf{Открытые множества} & \ \ \ \ \ \textbf{Замкнутые множества} \\
\hline
$\mathbb{R}$ & $\emptyset$ \\ \hline
$\emptyset $ & $\mathbb{R}$ \\ \hline
Произвольные объединения открытых & Произвольное пересечения замкнутых \\ \hline
Конечные пересечения открытых & Конечные объединения замкнутых \\ \hline
\end{tabular}

Одни свойства переходят в другие при взятии дополнения. Это не случайность.

\textbf{Теорема}. Множество открыто тогда и только тогда, когда дополнение до него замкнуто. Множество замкнуто тогда и только тогда, когда дополнение до него открыто. \\

\begin{proof}
Доказательство. Пусть $F$ — замкнутое подмножество прямой. Докажем, что $\mathbb{R} \setminus F$ открыто. Возьмём произвольную точку $x_0 \in \mathbb{R} \setminus F$. Эта точка не является предельной точкой $F$. Это значит, что найдётся её проколотая окрестность $\mathring{U}_{\varepsilon}(x_0)$, не пересекающаяся с $F$. Поскольку $x_0$ тоже не принадлежит $F$, вся окрестность $U_{\varepsilon}(x_0)$ не перескается с $F$, а значит, $x_0$ содержится в дополнении к $F$ вместе со своей $\varepsilon$-окрестностью. Следовательно, дополнение до $F$ открыто.\\ 

Пусть теперь $F$ — некоторое множество, дополнение до которого открыто. Докажем, что $F$ содержит все свои предельные точки. Рассмотрим точку $x_0 \in \mathbb{R} \setminus F$. По предположению у неё есть окрестность $U_{\varepsilon}(x_0)$, целиком содержащаяся в дополнении. Но тогда $U_{\varepsilon}(x_0) \cap F = \emptyset$, а потому и $\mathring{U_{\varepsilon}}(x_0) \cap F = \emptyset$. Значит, $x_0$ не может быть предельной точкой для $F$. Раз никакая точка, лежащая в дополнении к $F$, не является для него предельной, само множество $F$ содержит все свои предельные точки.\\
\end{proof}
Таким образом, достаточно было доказать только свойства открытых множеств — свойства замкнутых будут следовать из теоремы.\\

\textbf{Ещё о свойствах замкнутых множеств}\\ 
Открытые подмножества прямой описать довольно просто (проделайте это в качестве упражнения). По доказанной выше теореме замкнутые множества на прямой двойственны открытым, однако они, в отличие от открытых, могут быть устроены весьма сложно и не поддаются простой классификации. Тем не менее, замкнутые подмножества прямой обладают целым рядом замечательных свойств, например, таким: \\

\textbf{Теорема}. Для замкнутых подмножеств прямой верна \textit{континуум-гипотеза}: всякое бесконечное замкнутое подмножество прямой либо счётно, либо равномощно $\mathbb{R}$. \\

Напомним, что для произвольных подмножеств прямой \textit{континуум-гипотеза} независима от аксиоматики теории множеств.

\section{Всюду плотные и нигде не плотные множества}\\

\textbf{Замыкание} \\
\textbf{Определение 1}. Замыканием множества $A$ называется множество $\overline{A} := A \cup \{ x | x — \text{ предельная точка } A\}$. \\

\textbf{Утверждение 1}. $\overline{A}$ замкнуто. \\

\begin{proof}
Доказательство: Пусть $x$ является предельной точкой для $\overline{A}$. Докажем, что любая проколотая окрестность $\mathring{U}_{\varepsilon}(x)$ содержит точку из $A$. Действительно, рассмотрим $\mathring{U}_{\frac{\varepsilon}{2}}(x)$. Поскольку $x$ — предельная точка множества $\overline{A}$, то либо эта окрестность содержит точку из $A$ (а тогда и $\mathring{U}_{\varepsilon}(x)$ её содержит), либо эта окрестность содержит точку $y$, предельную для $A$. В этом случае в $\mathring{U}_{\frac{\varepsilon}{2}}(y)$ есть точка $a \in A$. Но по неравенству треугольника $a \in \mathring{U}_{\varepsilon}(x)$. \\

Следовательно, всякая предельная точка $\overline{A}$ является предельной также и для $A$. Но $\overline{A}$ содержит все предельные точки $A!$ Следовательно, $\overline{A}$ замкнуто.\\
\end{proof}
Более того, из доказательства утверждения видно, что верна следующая формула:
\begin{center}
$\overline{\overline{A}} = \overline{A}$
\end{center}

Действительно, множество предельных точек $\overline{A}$ совпадает со множеством предельных точек $A$, а значит содержится в $\overline{A}$. \\

Пример: замыкание интервала $(a, b)$ — это отрезок $[a,b]$. Замыкание $\mathbb{Q}$ — это вся действительная прямая. \\

Замыкание можно описать и по-другому. \\

\textbf{Определение 1}'. Замыканием множества $A$ называется пересечение всех замкнутых множеств, содержащих $A$.\\ 

\begin{proof}[Доказательство эквивалентности определений 1 и 1′]. Пусть $\overline{A}$ — замыкание $A$ в смысле определения 1, а $\widetilde{A}$ — замыкание $A$ в смысле определения 1′. Если $A \subseteq F$ для какого-то замкнутого множества $F$, то все предельные точки $A$ также являются предельными точками $F$, а поэтому лежат в $F$. Следовательно, $\overline{A} \subseteq \widetilde{A}$.\\

C другой стороны, $\overline{A}$ является замкнутым множеством, содержащим $A$, поэтому $\widetilde{A} \subseteq \overline{A}$. Получаем $\widetilde{A} = \overline{A}$.\\
\end{proof}

Можно задаться следующим вопросом: а существует ли, наоборот, наименьшее открытое множество, содержащее данное? В конструкции замыкания ключевую роль играл тот факт, что пересечение любого числа замкнутых множеств замкнуто. Для открытых множеств это неверно. Например, попробуйте разобраться, что произойдёт, если пересечь все открытые подмножества прямой, содержащие $\mathbb{Q}$. \\

\textbf{Всюду плотные и нигде не плотные множества 
}\\

\textbf{Определение 2}. Множество $A \subset \mathbb{R}$ называется всюду плотным, если в любом интервале $I \subset \mathbb{R}$ есть точка из $A$. \\

Эквивалентное определение.\\

\textbf{Определение 2}'. Множество $A$ всюду плотно, если $\overline{A} = \mathbb{R}$. \\

\begin{proof}[Доказательство эквивалентности определений 2 и 2′]. Пусть $A \subset \mathbb{R}$ — такое подмножество прямой, что любой интервал содержит точку из $A$. Докажем, что всякая точка $x \in \mathbb{R}$ является предельной для $A$. Рассмотрим произвольную проколотую окрестность $\mathring{U}_{\varepsilon}(x)$. Интервал $(x-\varepsilon;x) \subset \mathring{U}_{\varepsilon}(x)$ содержит точку $a \in A$, а значит $a \in \mathring{U}_{\varepsilon}(x)$. По определению 1 получаем $\overline{A} = \mathbb{R}$. 

В другую сторону. Пусть $\overline{A} = \mathbb{R}$ для множества $A \subset \mathbb{R}$. Возьмём произвольный интервал $(x_1, x_2) \subset \mathbb{R}$. Всякая точка $x \in (x_1, x_2)$ лежит в $\overline{A} = \mathbb{R}$, а потому либо принадлежит $A$, либо не принадлежит $A$, но является предельной для $A$. Если есть хотя бы одна точка $x \in (x_1, x_2)$, принадлежащая $A$, мы победили. В противном случае в интервале $(x_1, x_2)$ найдётся точка $x$, являющаяся предельной для $A$. Тогда во всякой её проколотой $\varepsilon$-окрестности есть элемент из $A$. Возьмём $\varepsilon$ настолько малым, что $\mathring{U}_{\varepsilon}(x) \subset (x_1, x_2)$. \\
\end{proof}
\textbf{Определение 3}. Множество $A$ называется нигде не плотным, если в каждом интервале $I \subset \mathbb{R}$ найдётся подынтервал $I’ \subseteq I$ такой, что $I’ \cap A = \emptyset$. 

Эквивалентное определение нигде не плотности. \\

\textbf{Определение 3}'. Множество $A$ называется нигде не плотным, если его замыкание не имеет внутренних точек.\\ 

\begin{proof}[Доказательство эквивалентности определений 3 и 3′]. Пусть $A \subset \mathbb{R}$ нигде не плотно в смысле определения 3. Предположим, что $x \in \overline{A}$ — внутренняя точка. Возьмём $\varepsilon>0$ такое, что $U_{\varepsilon}(x) \subseteq \overline{A}$. Мы утверждаем, что в интервале $I:= (x-\varepsilon, x+\varepsilon)$ нет подынтервала, не пересекающегося с $A$. Действительно, пусть $I’$ — такой подынтервал. Рассмотрим произвольную точку $x’ \in I’$. Можно найти такое $\delta>0$, что $U_{\delta}(x’) \subseteq I’$. Поскольку $I’ \bigcap A = \emptyset$, имеем $x’ \notin A$. Более того, $\mathring{U}_{\delta}(x’) \bigcap A = \emptyset$. Значит, $x’$ не является предельной точкой для $A$ и $x’ \notin \overline{A}$. Однако $x’ \in I’ \subset I \subset \overline{A}$. Противоречие. \\

В другую сторону. Пусть $A \subset \mathbb{R}$ нигде не плотно в смысле определения 3′. Предположим, что найдётся $I \subset \mathbb{R}$ интервал такой, что для всякого подынтервала $I’ \subset I$ пересечение $I’ \bigcap A$ не пусто. Докажем, что $I \subset \overline{A}$. Для любой точки $x \in I$ и её проколотой окрестности $\mathring{U_{\varepsilon}}(x)$ пересечение $\mathring{U_{\varepsilon}} \bigcap A \neq \varnothing$, поскольку $(x-\varepsilon; x) \bigcap A \neq \varnothing$. Cледовательно, любая точка $x \in I$ предельная для $A$, а значит содержится в $\overline{A}$. Но $I$ имеет внутренние точки, а значит и $\overline{A}$ имеет внутренние точки. Противоречие.\\
\end{proof}

Классический пример всюду плотного множества — рациональные числа. Классический пример нигде не плотного множества — канторово множество. \\

\textbf{Теорема Бэра}\\
С понятием нигде не плотного множества связана важная теорема, доказанная французским математиком Рене-Луи Бэром в 1899 году. \\

\textbf{Теорема (Бэр)}. Отрезок нельзя представить как счётное объединение нигде не плотных множеств. \\

Теорема Бэра имеет применения в математическом анализе (мы это увидим позже). Доказательство её совсем не трудно. \\

\begin{proof}[Доказательство теоремы Бэра]. Предположим обратное, то есть пусть $[0;1] = \bigcup_n A_n$ и каждое $A_n$ нигде не плотно. \\

Возьмём интервал $I_1 \subset [0;1]$, не содержащий ни одной точки $A_1,$ — такой найдётся по предположению о нигде не плотности $A_1$. Теперь возьмём какой-нибудь отрезок $S_1 \subset I_1$, а внутри $S_1$ выберем интервал $I_2$, не содержащий никакой точки $A_2$. Внутри $I_2$ возьмём отрезок $S_2$, в нём — интервал $I_3$, не содержащий точки $A_3$, и так далее. \\

В итоге получим последовательность вложенных отрезков $S_n$, причём для любого $n \geq k$ все отрезки $S_n$ лежат внутри интервала $I_k$, который не содержит ни одной точки множества $A_k$. В частноcти, $S_n \bigcap A_k = \emptyset$ при $n \geq k$. Из принципа вложенных отрезков пересечение $\bigcap_n S_n$ не пусто. Но любая точка, лежащая в этом пересечении, не принадлежит ни одному из множеств $A_n$. Мы пришли к противоречию.\\
\end{proof}

$F_{\sigma}$- и $G_{\delta}$-множества. \\
Как мы уже видели, объединение счётного числа замкнутых множеств не обязательно замкнуто, равно как и пересечение счётного числа открытых множеств не обязательно открыто. А какие множества можно получить такими конструкциями? \\

\textbf{Определение 4}. Подмножество прямой называется $F_{\sigma}$-множеством, если его можно представить как объединение счётного числа замкнутых множеств. \\

\textbf{Определение 5}. Подмножество прямой называется $G_{\delta}$-множеством, если его можно представить как пересечение счётного числа открытых. \\

Всякое счётное множество $X$ является $F_{\sigma}$-множеством, поскольку $X = \bigcup_{x \in X} \{x\}$. В частности, $\mathbb{Q}$ является $F_{\sigma}$-множеством. А является ли $\mathbb{Q}$ множеством типа $G_{\delta}$? \\

Ответ отрицательный. \\

Утверждение 2. $\mathbb{Q}$ не является $G_{\delta}$-множеством.\\

\begin{proof} Сперва заметим, что для любого открытого множества $U$, содержащего $\mathbb{Q}$, дополнение до него $\mathbb{R} \setminus U$ нигде не плотно. Действительно, возьмём любой интервал $I$ на прямой. Внутри него есть рациональное число $\frac{p}{q} \in \mathbb{Q} \subset U$. Поскольку $U$ открыто, оно содержит $\frac{p}{q}$ вместе с каким-то интервалом $I’ \subseteq I$. Значит, $I’$ не пересекается с $\mathbb{R} \setminus U$. \\

Теперь предположим, что $\mathbb{Q} = \bigcap_n U_n$ для какого-то набора открытых множеств $U_n$. \\

Дополнение $A_n \colon=\mathbb{R} \setminus U_n$ нигде не плотно для каждого $n$ и $\mathbb{R} \setminus \mathbb{Q} = \bigcup_n A_n$. Добавим к счётному набору нигде не плотных множеств $A_n$ ещё счётное количество нигде не плотных множеств, каждое из которых является просто одноточечным множеством {$\frac{p}{q}$}, а $\frac{p}{q}$ пробегает $\mathbb{Q}$. Тогда 
$\mathbb{R} = \mathbb{R} \setminus \mathbb{Q} \cup \mathbb{Q} = \bigcup_n A_n \cup \bigcup_{\frac{p}{q} \in \mathbb{Q}}$ {$\frac{p}{q}$} \\

Мы покрыли всю прямую счётным объединением нигде не плотных множеств. Это, конечно, противоречит теореме Бэра.\\
\end{proof}



\end{document}
